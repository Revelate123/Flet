\documentclass{article}%
\usepackage[T1]{fontenc}%
\usepackage[utf8]{inputenc}%
\usepackage{lmodern}%
\usepackage{textcomp}%
\usepackage{lastpage}%
\usepackage{geometry}%
\geometry{tmargin=1cm,lmargin=1cm}%
%
%
%
\begin{document}%
\normalsize%
\section*{Design of Concrete Column}%
\label{sec:DesignofConcreteColumn}%
\subsection*{Squash Load N_{uo}$ AS3600:2018 Cl 10.6.2.2}%
\label{subsec:SquashLoadNuoAS36002018Cl10.6.2.2}%
The ultimate strength in compression without bending (Nuo) shall be calculated by assuming:%
\newline%
\newline%
%
\begin{tabular}{ll}%
(a)&a uniform concrete compressive stress of $\alpha_{1}f'_{c}$ where:\\%
&$\alpha_{1} = 1.0 - 0.003f'_{c}$ with the limits 0.72 to 0.85 and\\%
&\\%
(b)&a maximum strain in the reinforcement of 0.0025.\\%
\end{tabular}

%
\subsection*{Calculation of Squash Load N_{uo}}%
\label{subsec:CalculationofSquashLoadNuo}%
$N_{uo} = \alpha_{1} f'_{c} A_{g} + A_{s} f_{sy}$%
\newline%
\newline%
%
$\alpha_{1} = 0.925$%
\newline%
\newline%
%
The section is circular, therefore:%
\newline%
%
$Ag = \pi r^{2}$%
\newline%
\newline%
%
$A_{st} = $1885.0$mm^{2}$%
\newline%
\newline%
%
$Ag = 70686.$ $ mm^{2}$%
\newline%
\newline%
%
$N_{uo} = 2577.1 $ kN%
\newline%
\newline%

%
\subsection*{Decompression Point}%
\label{subsec:DecompressionPoint}%
\subsection*{Decompression point AS3600:2018 Cl 10.6.2.3}%
\label{subsec:DecompressionpointAS36002018Cl10.6.2.3}%
The decompression point is calculated taking the strain in the extreme compressive fibre equal to 0.003, the strain in the extreme tensile fibre equal to zero and using the rectangular stressblock given in Clause 10.6.2.5

%
\subsection*{Calculation of Decompression Point}%
\label{subsec:CalculationofDecompressionPoint}%
Hello

%
\end{document}